\documentclass[fontsize=12pt]{scrartcl}

\usepackage[a4paper]{geometry}

\usepackage{ngerman}

\begin{document}
	{\huge Zwischen den Verpflichtungen}
	\vspace{15pt}
	
	{\LARGE D}ie Tür fiel schwer ins Schloss und  im Kerker kehrte endlich Stille ein. Severus, der noch kerzengerade in der Mitte vor den Bänken stand, in denen bis eben noch geschwätzige, unbelehrbare DrittklässlerInnen versucht hatten, minderwertigen Krötensud herzustellen. 
	
	Nach einem tiefen Atemzug, der ihn von den Unzulänglichkeiten der SchülerInnen ablenken sollte, wandte er sich seinem eigenen Zaubertrank zu, den er seit einigen Wochen braute. 
	
	Ein kurzer Stoß mit dem Zauberstab entfachte das Feuer, ein Schwenk brachte den Kessel darüber in Stellung. 
	Die trübe, zähflüssige, nach Ammoniak stinkende Brühe waberte träge gegen den Rand des Kessels, als sich die Ton-Amphore darin ergoss.
	Sich langsam erinnernd, rührte Severus gemächlich um und nahm die Qualität des Trankes in Augenschein. 
	Er sog die ruhige Geräuschkulisse des knackenden Feuers und des blubbernden Kessels in sich auf. 
	Er mochte diesen Moment, in dem sich die SchülerInnen gerade verzogen hatten und er mit sich wieder alleine war. 
	
	Das Schönste am Brauen, fand Severus, war die Kombination aus handwerklichem Geschick, methodischer Präzision, und nicht zu unterschätzender Kreativität. 
	Und man brauchte Mut, dachte sich Severus. 
	Nicht nur, dass mancher Brauvorgang gefährlich werden konnte. Die wahren Meister ihres Fachs blieben nicht auf den ausgetretenen Wegen, sondern erfanden Neue. 
	Versuchten Zutaten und Methoden zu kombinieren, die noch nicht kombiniert worden waren, um neuere, mächtigere Tränke hervorzubringen oder Altbewährte zu verbessern. 
	
	Ein bisschen Wermut, ein kleines Stückchen des geriebenen Bowtruckle Zehs, und den Rhythmus beim Umrühren nicht unterbrechen. Zweimal mit dem Uhrzeigersinn, dreimal dagegen, ohne Pausen. 
	Sonst war die ganze Arbeit der letzten Wochen umsonst.
	
	Plötzlich ein Stechen um linken Unterarm. Ein Brennen, dass ihn fast dazu gebracht hätte, das Rühren zu unterbrechen. Aber nur fast. 
	Denn Severus hatte dieses Stechen schon zu oft gespürt, um sich davon einen so aufwendigen Trank zerstören zu lassen. Dennoch war seine Konzentration gestört. 
	Denn das Stechen kam von seinem Mal, vom Ruf des dunklen Lords. Während Severus sich bemühte den Trank schnell, aber korrekt zu verstauen, machten sich düstere Gedanken breit in seinem Kopf. 
	Was würde diesmal der Grund sein? Standen große Pläne bereit zur Ausführung, oder musste jemand bestraft werden, weil ein kleiner Teil des großen Plans nicht funktioniert hatte? Wahrscheinlich beides.
	
	Ein großer Schwung seines Zauberstabes ließ eine silberne Hirschkuh erscheinen und mit der Nachricht an Dumbledore, dass er, Severus, gerufen wurde und dem Ruf folgen würde, in Richtung des Schulleiters davon preschen.
	
	Nun musste Severus zur inneren Ruhe kommen. Dies war essenziell um den kalten, starrenden Augen des dunklen Lords standhalten zu können. 
	Legilimenthik war eine, vielleicht sogar die bedeutendste Fähigkeit Voldemorts, auf seinem Weg zur Macht. Ein kurzes Schütteln des Kopfs, um die Gedanken an das Gefühl des Geröntgt-Werdens los zu werden, gelang Severus nur halb.
	Seine Konzentration galt nun nur diesem einen Moment, der für ihn schon immer am besten geeignet war, um sich von der Welt abzuwenden, sich komplett in sich kehren zu können. 
	
	
	Er war wieder ein Junge, der als Sprössling einer alten Zaubererfamilie in einer Muggelgegend aufwuchs. Er spürt die Hitze und Schwüle des Sommers Anfang August, nur wenig unterbrochen von leichtem und unstetem Wind.
	Auf dem Spielplatz unweit seines Elternhauses stand er neben der Schaukel, auf der Lily Evans saß. Sie waren weit und breit alleine. Sie hatte ihn gerade wieder gelöchert mit Fragen über die Zaubererwelt.
	
	Dann hatte einen lang Moment Stille geherrscht, in dem sie beide nachdachten und die Ruhe genossen. Einer Eingebung folgend bemerkte Severus, dass sie ja nicht in einer Muggelgegend bleiben müssten, sondern es auch Zauberergemeinden geben würde. 
	
	Es viel ihm im selben Augenblick, als er zu Ende gesprochen hatte auf, dass er 'wir' gesagt hatte. 'Wir müssen nicht in dieser Muggelgegend bleiben'. 
	Lily ging es wohl genauso, denn sie drehte ihren Kopf, der zuvor gedankenverloren zu Boden gewandt war, ruckartig zu ihm um. Ihre roten, strähnigen Haare wirbelten noch um ihren überaus hübschen Kopf, als ihre klugen, grünen Augen ihn fokussierten.
	
	Ihr Blick traf seinen und Severus spürte, dass Lily die Gefühle, die er selbst noch nicht verstand, in seinem Blick und seinen Gedanken lesen konnte. 
	Und Severus spürte, dass sie diesen Gedanken nicht sofort verwarf, ja vielleicht sogar in Erwägung zog, oder an ihm Gefallen fand. 
	Nach einem wunderschönen und doch auch unheimlichen Moment, wandte Lily ihren Blick wieder ab. Severus fühlte sich seltsam. 
	Lily antwortete etwas ausweichendes, was er nicht für schlechte Neuigkeiten hielt, aber auch unbefriedigend war. 
	
	Severus kam zurück in seinen Kerker in dem Moment, als er seine Augen wieder aufschlug. Nun war er bereit dem dunklen Lord gegenüber zu treten, befragt zu werden und dem bohrenden Blick standzuhalten. 
	
	Dieser Moment, in dem sich sein Blick und die Augen von Lily trafen, der Moment, in dem sich all die Gefühle in seinem Inneren Bahn brachen, dieser Moment würde ihm die Stärke geben, alles durchzustehen, was ihm bevorstand. 
	Das wusste Severus. Er wusste es aus tiefstem Herzen, so wie er nun seine Gefühle von damals deuten konnte. Er hatte das Geühl, viel falsch gemacht zu haben in seinem Leben. Nun war eine weitere Chance, ein kleines Stückchen wiedergutzumachen. 
	Auch wenn Severus genau wusste, dass er sich niemals alles würde verzeihen können. 
	
	Mit den klugen, grünen Augen eingebrannt in seinem Geist, brach Severus auf um sich ein weiteres Mal seiner Vergangenheit und seinen Fehlern zu stellen. 
\end{document}