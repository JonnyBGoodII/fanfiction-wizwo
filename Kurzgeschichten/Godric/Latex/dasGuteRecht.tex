\documentclass[fontsize=12pt]{scrartcl}

\usepackage[a4paper]{geometry}

\usepackage{ngerman}

\begin{document}
	{\huge  Das gute Recht}
	\vspace{15pt}
	
	{\LARGE G}odric warf noch einen Eichenscheit in die Glut, die laut knisterte und sich sogleich begierig über den Nachschub hermachte. Goldene Flammen beleuchteten den Raum als sich der Hogwartsgründer noch ein zweites Glas einschenkte von dem teuflisch guten Wasser. Die Leute nannten es nun \lq Whyskey\rq, dieses Gebräu aus Malz, Wasser, und ein paar anderen Zutaten, die Godric sich nicht merken wollte.
	
	Der Sessel war zum Kamin gewandt, sehr weich und gemütlich, obwohl er langsam etwas durchgesessen war, befand Godric, als er sich darauf niederließ. Der Sumpfkrattler, den er vor vielen Jahren in eben diesen Sessel verwandelt hatte, war auch nicht angenehm gewesen. Sonst hätte er ihn schließlich nicht verwandelt.
	
	Der Blick wanderte vom Feuer weg, im Raum umher, und blieb bei seinem meisterhaft geschmiedeten Schwert hängen, das über dem Kamin hing. Seine Stimmung verdüsterte sich, als er an die Kobolde dachte, die vor einiger Zeit des Schwertes wegen hier gewesen waren. Vielleicht hätte er die Delegation auch in Möbel verwandeln sollen, dann hätte er sich einigen Ärger ersparen können und der Astronomieturm hätte ein paar Tische und Regale bekommen.  
	
	Ein tiefer Seufzer folgte dieser Überlegung. Natürlich hatte er das nicht gekonnt. Kobolde waren schließlich sehr intelligente, fühlende Geschöpfe, anders als Sumpfkrattler, die schon mal eine halbe Stunde lang auf einem Wanderschuh, der im Sumpf stecken geblieben war, herumbeißen konnten, ehe sie merkten, dass darin keine Beute mehr war. Kobolde hingegen waren schlau und vor allem konnten sie hinterlistig sein. 
	
	Die Kobolddelegation und der Streit, den er mit den Dreien hatte, beschäftigte Godric schwer. Grisnak, Helask, und Kaledioth waren an der Pforte von Hogwarts aufgetaucht, und verlangten den Gründer Godric, den man nun Gryffindor nannte, zu sprechen. 
	
	Das Gespräch hatte bereits kühl begonnen. Schließlich hatten die Gäste mit den knorrigen Gesichtszügen, die so sehr zu den Charakteren gepasst hatten, fast zwei Stunden warten müssen. Nun, dachte Godric, es sei nicht anders gegangen. Denn sie wären unangekündigt gekommen und er habe sich als Professor der Ausbildung und Erziehung der nachsten Hexen- und Zauberergenerationen verpflichtet. Die Kobolde hatten hier wohl andere Prioritäten gehabt.
	
	Helask, die einzige weibliche Koboldin unter ihnen, war sofort zum Punkt gekommen. Sie hätte gerne das Schwert, das dort über dem Kamin hänge. Das hatte Godric in der Tat erst einmal verblüfft. Dort hing das Schwert, das er vor vielen Jahren von einem Kobold namens Merloth hatte fertigen lassen, den er auf Wanderschaft kennengelernt hatte. Sie hatten sich ganz gut verstanden und in den wenigen Wochen der gemeinsamen Reise gegenseitig geholfen und vieles gelernt. Bei einer Herausforderung durch einen Muggel hatte Godric mit einem geliehenen, von Muggeln geschmiedeten Schwert kämpfen müssen und trotz seiner herausragenden Fertigkeiten fast verloren. Daraufhin hatte sein Begleiter beschlossen, dass Goddric ein eigenes, standesgemäßes Schwert besitzen müsse. 
	
	Es hatte sich später herausgestellt, dass Merloth, mit dem er sich das Lagerfeuer geteilt hatte, ein wahrer Meister des Waffenschmiedens werden sollte. Godrics Exemplar war nur der Anfang einer Reihe exzellenter, herausragender Waffen aus seiner Hand geworden, wenn auch das einzige Schwert, das er je hergestellt hatte. 
	
	Unter Kobolden war man der Auffassung, dass das SchwertEigentum nur solange in Besitz des Käufers blieb, bis dieser es nicht mehr benötigte. Danach ging es wieder an den Erzeuger zurück. Hier also an den Schmied oder besser: dessen Erben. 
	
	Genau hier waren die beiden Streitpunkte, die für die Auseinandersetzung ausschlaggebend gewesen waren. Godric war sich sicher das Schwert noch mehr als einmal gebrauchen zu können. Und wenn nicht er selbst, dann jemand, für den er verantwortlich war. Es wäre schrecklich zu wissen, dass das Schwert einem seiner SchülerInnen helfen konnte, während es in irgendeinem Kobold Verlies herumlag. Das musste Grodric verhindern! Er musste einstehen für die, die es nicht so konnten wie er. 
	
	Außerdem teilte er die Auffassung nicht, das Schwert je wieder zurückgeben zu müssen. Er hatte sich die Freundschaft des Schmiedes durch Taten und Beistand verdient und letztendlich trotz der Freundschaft einen stolzen Preis in Gold bezahlt. Es war und blieb sein Schwert. Wenn die Kobolde dies nicht akzeptieren wollten, dann sollten sie keine Geschäfte mit den Menschen machen. Pah!
	
	Die Debatte über diese beiden Punkte war so müßig wie aussichtslos für beide Parteien, da Godric einen Dickschädel hatte, der dem eines Trolls nahe kam. Selbst die Kobolde, die für ihre Sturheit bekannt waren, mussten überrascht gewesen sein. 
	
	Die Argumente zu Anfang waren sogar einen Gedanken wert gewesen: Ein Gründer und Professor eines Internats für Hexen und Zauberer umgab sich nur mit zauberfähigen Wesen, weshalb er kein Schwert für Duelle mit Nicht-zauberfähigen mehr benötige, befand Grisnak, der Älteste der Kobolde. Wenn er ehrlich zu sich selbst war, hatte er nicht vor den wenigen Kontakt, den er noch hatte zu den Muggeln, auszubauen. 
	
	Godric hatte gelernt, dass Ausweichen manchmal besser war als der Gegenangriff, den er liebte. Doch hier hatte er wirklich diplomatisch reagieren wollen, und hatte ausweichend geantwortet. Man wisse ja nie, wann es wieder zu Duelle kam, et cetera. Das hatte allerdings auch die Kobolde nicht befriedigt. Erschwerend kam der Whyskey hinzu, den er eigentlich aus Gastfreundschaft ausgoss, der aber die Gemüter eher befeuerte als sie zu beruhigen. Das war nicht seine beste Idee an diesem Tag gewesen befand Godric rückblickend. 
	
	Kaledioth hatte offensichtlich irgendwann genug und änderte die Taktik. Statt mit sachlichen Argumenten zu überzeugen, wollte der jüngste Kobold Godric nun bei seiner Ehre packen. Er behauptete, das Schwert sei von solcher Schönheit und Güte, dass es nur einem wahrer Meister zustünde. Ob Godric Gryffindor dem nicht zustimme? Das konnte er um nichts in der Welt verneinen. Allerdings witterte Godric hier einen Hinterhalt, den Kaledioth trotz seiner Jugend hier legen wollte. Dies machte den ehrwürdigen Gründer wütend, denn ihn konnte man am leichtesten in Rage versetzen, wenn man ihm das Gefühl gab, übers Ohr gehauen zu werden. 
	
	Natürlich kam sofort die Herausforderung von Kaledioth an Godric um das Schwert. Doch nun reichte es wirklich. Er hatte sie freundlich empfangen, sich gastfreundlich gezeigt, sogar argumentieren wollen, wo es eigentlich nichts zu diskutieren gab. Und nun per List aufs Kreuz gelegt werden! Ohnehin hätte er sich nicht auf das Duell einlassen müssen, da Kobolde sehr wohl magische Wesen waren und mit anderen Mitteln kämpfen konnten als mit Schwertern. Außerdem war der Grund für das Duell es nicht wert gewesen, da er nicht aufrichtig war. Das hatte Godric sehr wohl erkannt, als er den dreien nun tief in die Augen blickte. Ein Duell musste ehrenhaft sein, also aus einem ehrenhaften Grund und mit ehrenhaften Mitteln geführt werden. All dies hatten sie hier verletzt, diese Schufte. 
	
	Auf den Hinauswurf von Godric reagierten die Kobolde sehr gereizt, man wurde immer lauter. Selbst hier, erkannte Godric nun, lag immer noch List und Tücke von Seiten der Kobolde vor, denn sie versuchten nun unverhohlen ihn zu provozieren. Eine Beleidigung war nun einmal schon ein besserer Anlass für ein Duell, das wussten beide Seiten. 
	
	Godrics Groll war übermächtig geworden. Er hatte sich vor den drei mickrigen, wütend schnaubenden Gestalten aufgebaut, und ein für alle mal klargestellt, wer der eigentliche und rechtmäßige Besitzer des Schwertes war. Hatte gerade heraus gesagt, dass mit List und Tücke hier nichts zu erreichen war. Als dann auch noch einer der drei, Godric erinnerte sich nicht wer es gewesen war, die Anschuldigung dreist von sich weisen wollte, platze ihm der Kragen. Er warf sich wutschnaubend und mit einer Drohung hinaus. Sollten sie noch einmal hier auftauchen und mit derlei hanebüchenen Geschichten versuchen ihn zu betrügen, dann würde er über sie alle kommen und sich rächen. Ein paar Funken und etwas Getöse mit dem Zauberstab untermalten seine Absichten.
	
	Das hatte die gewünschte Wirkung erreicht, die Kobolde suchten mit wehenden Gewändern das Weite. Godric brauchte einige Zeit um sich zu beruhigen. Nun, mit etwas Abstand, dachte er etwas anders über die Dinge. Aber was hätte er auch anderes machen sollen? Er konnte das Schwert nicht zurückgeben, nur weil es deren Brauch war. Es hatte hier in Hogwarts noch viele wichtige Dinge zu vollbringen. Und ein Duell mit einem Kobold war bestimmt keine gute Idee. 
	
	Das Glas war leer, der rauchig, torfige Duft des Gebräus lag noch darin. Auch die Glut war dunkelrot geworden und spendete nur wenig Licht. Die Beziehungen zwischen Menschen und Kobolden waren sicherlich nicht besser geworden in letzter Zeit. Allerdings lag das nicht direkt an ihrer Auseinandersetzung, sonder viel mehr an unterschiedlichem Empfinden über Richtig und Falsch. Es hatte früher oder später zu solchen Disputen kommen müssen. Und es werde wohl auch immer wieder zu derlei Uneinigkeit kommen...
	
	\end{document}
