\documentclass[fontsize=12pt]{scrartcl}

\usepackage[a4paper]{geometry}

\usepackage{ngerman}

\begin{document}
	{\huge  Freunde}
	\vspace{15pt}
	
	{\LARGE H}elga schnalzte unzufrieden mit der Zunge. Es wollten ihr keine guten Ideen einfallen. Einerseits sollten natürlich die Geschmäcker der GründerInnen getroffen werden, aber auch die der SchülerInnen und der Angestellten, die nun mal nicht nur Menschen waren. Da waren auch ein Schatzmeister, der ein sehr listiger Kobold sein konnte, ein Zentaur, der sich zumindest hin und wieder meldete, wenn es um Angelegenheiten des naheliegenden Waldes ging, und der aus eher diplomatischen Gründen eingeladen war. Außerdem ...
	
	Das alles wollte versorgt werden. Und gute Stimmung konnte man mit einem ausgezeichneten Essen zwar nicht erzwingen, aber auf jeden Fall zerstören. Das war ihr klar. Auf die Stimmung würde es ankommen, bei der Jubiläumsfeier von Hogwarts. Denn es würden wohl auch weniger gute Erinnerungen, dachte sich Helga. Schließlich wurde die Gründung einer Schule gefeiert, bei der sich ein Gründer schon freiwillig und nicht im Guten verabschiedet hatte. 
	Das tat immer noch weh, stellte sie fest. Sie waren ein so tolles Quartett gewesen. Godric mit dem unerschütterlichen Mut, Rowena mit der unerreichten Gedankenschnelligkeit, und Salazar, der immer noch einen Weg gefunden hatte, wenn es eigentlich schon zu spät war. Und sie selbst, die sich einfach verantwortlich fühlte, das die Gruppe funktionierte. Man musste ihnen immer irgendwie helfen, auch wenn sie als die größten MagierIn ihrer Zeit galten. Sie konnten sich unheimlich selbst im Wege stehen. Aber das konnte Helga schon immer gut. Die Leute zusammen und bei Laune halten, zum Beispiel, wenn es um kleine Streitigkeiten ging, konnte sie vermitteln, oder eben durch ein gutes Essen die Stimmung heben und ein großes Fest erst ermöglichen. 
	Nur das eine mal mit Salazar und Godric wollte es nicht klappen, und das hatte schwer wiegende Konsequenzen gehabt. Das lag auch den drei verbliebenen GründerInnen auf der Seele, das wusste sie ganz genau. 
	Doch nun zurück zu den Freuden des Lebens, dachte sie sich, und fing eine neue Süßspeise an, mit Zutaten, die sie noch nie kombinierte hatte: Haferflocken, einen großen Kleks Butter, Gojibeeren, Heidelbeeren, Honig, Flohsamenschalen, gehackte und geröstete Mandeln. Dazu eine Erdbeercreme mediteran abgeschmeckt. 
	Das erhellte ihre Stimmun sofort, und es ging mit neuem Eifer an die Küchenplanung. Alles sollten sie glücklich und zufrieden an den Festbänken verweilen, wenn schon nicht wegen der Gesellschaft, dann doch wenigstens, weil man sich überfressen hatte.
	
	\end{document}