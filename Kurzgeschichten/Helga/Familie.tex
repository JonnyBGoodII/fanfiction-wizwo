\documentclass[fontsize=12pt]{scrartcl}

\usepackage[a4paper]{geometry}

\usepackage{ngerman}

\begin{document}
	{\huge  Familie}
	\vspace{15pt}
	
	{\LARGE H}elga schnalzte unzufrieden mit der Zunge. Es wollten ihr einfach keine guten Ideen einfallen. Einerseits sollten natürlich die Geschmäcker der GründerInnen getroffen werden, aber auch die der SchülerInnen und der Angestellten, die nun mal nicht nur Menschen waren. 
	
	Es gab auch einen Schatzmeister, der ein sehr listiger Kobold sein konnte, einen Zentaur, der sich zumindest hin und wieder meldete, wenn es um Angelegenheiten des naheliegenden Waldes ging. Er wurde aus eher diplomatischen Gründen eingeladen. 
	
	Außerdem noch Bogdan, der gutaussehende Vampir aus Transsylvanien. Er unterrichtete Geschichte der Magie. Hier machte Helga sich schon mal eine Notiz, einige Speisen ohne Knoblauch zu servieren. 
	
	Und natürlich noch Veluria die Veela, von der niemand so recht wusste, woher sie kam. Aber in Verteidigung gegen andere Magier, war sie unschlagbar. Vielleicht half aber auch ihr Anblick beim ein oder anderen Gegner... 
	
	Das war ihr noch nie passiert, stellte Helga ernüchtert fest. Aber es zählten andere Werte, wie Kameradschaft, Treue, gegenseitige Unterstützung, und Verständnis füreinander. Hier konnte ihr niemand etwas vor machen, schon gar nicht die unnahbare Veluria. 
	
	Jedenfalls wollte das alles versorgt werden. Und gute Stimmung konnte man mit einem ausgezeichneten Essen zwar nicht erzwingen, aber mit einem schlechten, konnte man sie auf jeden Fall zerstören. Das war ihr klar. Auf die Stimmung würde es ankommen, bei der Jubiläumsfeier von Hogwarts. Denn es würden wohl auch weniger gute Erinnerungen hervorgebracht werden, dachte sich Helga. Schließlich wurde die Gründung einer Schule gefeiert, bei der sich ein Gründer schon freiwillig und nicht im Guten verabschiedet hatte. 
	
	Das tat immer noch weh, stellte sie fest. Sie waren ein so tolles Quartett gewesen. Godric mit dem unerschütterlichen Mut, Rowena mit der unerreichten Gedankenschnelligkeit, und Salazar, der immer noch einen Weg gefunden hatte, wenn es eigentlich schon zu spät war. Und sie selbst, die sich einfach verantwortlich fühlte, dass die Gruppe funktionierte. 
	
	Man musste ihnen immer irgendwie helfen, auch wenn sie als die größten MagierIn ihrer Zeit galten. Sie konnten sich unheimlich selbst im Wege stehen. Aber da musste Helga schon immer helfen. Die Leute zusammen und bei Laune halten, zum Beispiel, wenn es um kleine Streitigkeiten ging, versuchte sie zu vermitteln, oder eben durch ein gutes Essen die Stimmung heben und ein großes Fest erst ermöglichen. 
	
	Nur das eine mal mit Salazar und Godric wollte es nicht klappen, und das hatte schwer wiegende Konsequenzen gehabt. Das lag auch den drei verbliebenen GründerInnen auf der Seele, das wusste sie ganz genau. Ach, wenn man die Zeit zurückdrehen könnte! Oder einfach nur wieder miteinander reden würde. Mit dem Reden könnten sie sich annähern. Aber diese sturen Esel waren wohl diesem unsäglichen Mythos von Ehre und Rechthaberei verfallen, so dass keiner den ersten Schritt machen konnte... 
	
	Doch nun zurück zu den Freuden des Lebens, dachte sie sich, und fing an, etwas neues zu kreieren. Das würde sie auf andere, heiterere Gedanken bringen. Eine neue Süßspeise hörte sich gut an. Mit Zutaten, die sie noch nie kombinierte hatte: Haferflocken, einen großen Klecks Butter, Gojibeeren, Heidelbeeren, Honig, gehackte und geröstete Mandeln. Dazu eine Erdbeercreme eher südländisch abgeschmeckt. 
	
	Das erhellte ihre Stimmung sofort, und es ging mit neuem Eifer an die Küchenplanung. Alles sollten sie glücklich und zufrieden an den Festbänken verweilen, wenn schon nicht wegen der Gesellschaft, dann doch wenigstens, weil man sich überfressen hatte.
	\end{document}